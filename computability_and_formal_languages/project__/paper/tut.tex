\documentclass[11pt]{article}

\author{J Bolton}
\title{A Comparitive Exploration of Three Unusual Languages}
\frenchspacing
\pagestyle{headings}

\begin{document}

\maketitle

\tableofcontents

\begin{abstract}
    This paper provides an introductory tutorial to three contrasting and
    unusual programming languages: Smalltalk, SETL, and Agda. Some prior
    experience in programming is assumed. The languages are highly abstract yet
    represent very different perspectives: the object oriented paradigm,
    functional programming with dependent types, and set-based imperative
    programming. Working with each of them at the same time will be an
    invigorating brain exercise and will hopefully provide some unique insights
    into the strengths and weaknesses of each style.
\end{abstract}

\section{Introduction}
    Agda represents the extreme end of functional type theory based on the
    elegant syntax of Haskell. The language will introduce the very
    interesting world of dependent types and all the expressiveness that comes
    with them. Smalltalk is a much older language known for its pure
    implementation of the object oriented type system, made famous by such
    extremely popular languages as C++ and Java. Finally, SETL is the oldest of 
    the three languages whose main feature is its flexible, built-in support 
    for creating and manipulating sets and tuples.

    We will start out with a very quick overview of each language,
    starting with the least difficult, SETL, and progressing to the most
    difficult, Agda. This paper will emphasize a discussion of the typing
    model of each language.

    We will then examine three programs chosen to showcase the advantages of
    each of the languages. We will follow this with a larger, single program
    (an LL parser generator) written in all three languages to provide a
    direct, contrasting example among all three styles.  The larger program, a
    top down parser, was chosen in the hope that none of the three paradigms
    would have a very big advantage in its implementation over the others 

\section{SETL}

\subsection{Squishing Trees of Numbers}

SETL uses an imperative style of programming similar to the classic language
Ada. The language has assignments, loops, conditionals, and arithmetic that
shouldn't look all that foreign to those familiar with other imperative
languages. 

We will examine the language by walking through a couple problems that seem
to particularly suit its style. We will end by examining a general problem
that we will also implement in Smalltalk and Agda later on. 

The first of our problems comes from the puzzle website called Project
Euler. It involves finding a maximal path through a large tree of random
numbers. These aren't quite binary trees, but more like lattices, where every
child node is shared to the left and right. This is the statement of the problem
from the website:

\begin{quote}
By starting at the top of the triangle below and moving to adjacent numbers on
the row below, the maximum total from top to bottom is 23.

\begin{verbatim}
   3
  7 4
 2 4 6
8 5 9 3
\end{verbatim}

That is, 3 + 7 + 4 + 9 = 23.

Find the maximum total from top to bottom in [a triangle with 100 levels].
\end{quote}

The strategy is to select the second highest level and `collapse' each node by
finding the largest adjacent node above it. Eventually, we will produce a
single list consisting of the bottom row where each node is the maximum sum of
some path that led to it.

This problem is made simpler by the fact that we only need to find the maximum
sum, and we don't have to list the path. 

\begin{verbatim}
   3
  7 4
 2 4 6
8 5 9 3
First, we select the row [7,4], and collapse the top level to make:
  10 7
 2  4 6
8  5 9 3
Now select the row [2,4,6] and collapse it with the top level again. 
In the case of the middle node, 4, choose its maximum adjacent node (10).
 12  14 13 
8  5   9  3
A final time:
20 19 23 16
Now, we simply need to fold the two-ary max function over this final list to
get our answer, which is 23.
\end{verbatim}

Let's translate this to SETL. Our first step is to get our tree into 
the language. We can use SETL's read function, which takes a string of a certain
format and returns a corresponding data type. For example, if we give it...

\begin{verbatim}
[[3]
[7 4]
[2 4 6]
[8 5 9 3]]
\end{verbatim}

... we will get a tuple of tuples. In SETL, tuples are represented by brackets and
can be heterogenous, but ours will contain only numbers. Tuples differ
from sets in that they can contain duplicates and follow a specific sequence.

We will need to iterate over our tree, which is a tuple of tuples. We
know we're done when there is only one level left in the tree. Our
loop can then look like:

\begin{verbatim}
loop while #Tree /= 1
    ...
end loop;
\end{verbatim}

The pound sign is the unary function 'length of'. Nice and terse, isn't it?
Inside this loop, we need to manipulate the level that is second to the top, as
planned. In SETL, we have really handy subtuple extraction similar to many other
languages by simply inserting an index (or a range of indices) inside a pair of
parens after the tuple. \begin{verbatim}Tree(1)\end{verbatim} will return the
top level of our tree, while Tree(2) will return the second to the top, as we
need.

When we have a tuple of tuples, we can layer these subtuple extractions.
\begin{verbatim}Tree(2)(1)\end{verbatim} will return the first node of the
second level of our tree. Tree(2)(\#Tree(2)) will return the last node of the
second level of our tree. In our loop, we will define what happens to the
leftmost and rightmost nodes in the level first, since they are a special
case. For all the nodes in the middle, we have to find the maximum of the two
nodes that are above and adjacent to the them. For nodes at the very beginning
and end, we only need to add the node above, since they have only one adjacent
node.

\begin{verbatim}
  loop while \#Tree /= 1
      tree(2)(1) +:= tree(1)(1);
      tree(2)(#tree(2)) +:= tree(1)(#tree(1));
  end loop;  
\end{verbatim}

Next, we need to loop through the nodes in the middle, adding the maximum of
their adjacent nodes. In SETL, we can create lists with the form
\begin{verbatim}
[start..end]
\end{verbatim}

We need to iterate from the second node in our level to the second to last
node. 
\begin{verbatim}
[2..#tree(2)-1]
\end{verbatim}
will do just the thing. Let's put the loop together:
\begin{verbatim}
  (for n in [2..#tree(2)-1])
    tree(2)(n) +:= tree(1)(n) max tree(1)(n-1);
  end;
\end{verbatim}

'max' is an infix binary operator. The head of the loop could also be written as
'loop for n in...', but surrounding it by parentheses means you can get rid of
the 'loop' keyword.

All that's left to do is to remove the top row. Both sets and tuples in SETL are
mutable, so you can remove stuff, add stuff, and monkey with them as much as you
please. The binary operator 'fromb' will remove an element from the beginning of
a tuple. \footnote{'fromb' requires a left hand side variable to assign
  our discarded level to. We don't need to do anything with that variable,
  however, so it would be cleaner if we didn't do any assignment, but I have not
  seen any such function built in.} Let's put it all together.

\begin{verbatim}
read(Tree);
(while #Tree /= 1)
  tree(2)(1) +:= tree(2-1)(1);
  tree(2)(#tree(2)) +:= tree(1)(#tree(1));
  (for n in [2..#tree(2)-1])
    tree(2)(n) +:= tree(1)(n) max tree(1)(n-1);
  end;
  q fromb tree;
end;
\end{verbatim}

We have one last step. After the above loop finishes, we will be left
with a single tuple. Now we need to fold 'max' over our final
tuple. Fold in SETL is the wonderfully terse forward slash character.

\begin{verbatim}
print(max/Tree(1));
\end{verbatim}

\subsection{Best Friend Numbers}

Let's work through another problem, this time emphasizing \textit{set
  and tuple generation} using SETL's wonderful built-in syntax. This
next problem also comes from the wonderfully addictive Project
Euler. I will let the website do the explaining for me:

\begin{quote}
Let d(n) be defined as the sum of proper divisors of n (numbers less
than n which divide evenly into n).

If d(a) = b and d(b) = a, where a ≠ b, then a and b are an amicable
pair and each of a and b are called amicable numbers.

For example, the proper divisors of 220 are 1, 2, 4, 5, 10, 11, 20,
22, 44, 55 and 110; therefore d(220) = 284. The proper divisors of 284
are 1, 2, 4, 71 and 142; so d(284) = 220.

Evaluate the sum of all the amicable numbers under 10000.
\end{quote}

The algorithm for this problem is a little more straighforward than the
last one. Let's first tackle the 'proper divisors' function.

The set generating syntax in SETL is the same as what is now known to
be called `list comprehension' in Haskell. Functions in GNU SETL are
denoted by the 'proc' keyword and take untyped parameters.

We'll start with a naive version of the proper divisors function:

\begin{verbatim}
proc properdivisors(n);
     return {x : x in [1..n] | n mod x = 0};
end proc;
\end{verbatim}

The curly braces denote a set. Before the colon, we have x, which is
going to be the format of the elements in our set. This could be an
expression or tuple as well. 'x in [1..n]' denotes an iterator that
repeatedly assigns x to each value in the tuple [1..n]. After the bar,
we have a boolean expression which can specify constraints on our
variables. 

This is a very brute force way of finding the divisors of n. Since
we're going to have to do this thousands and thousands of times, let's
clean it up a bit. We'll use an algorithm called the Sieve of
Eratosthenes, the details of which I won't cover here. There is an
optimized version called the Sieve of Atkin, but we're only working
with numbers less than 10,000, so we might as well keep it simple.



\subsection{Set Magic}

% This section will hold a tutorial on all of SETL's set, tuple, and
% map generators and operations. This is the meat of the language.

\subsection{SETL's Type Model}

% I may need to do more work here, reading up on classes and such. 
% Emphasize the fact that SETL is all about maps. Use the quote from Dewar.

\subsection{Case Study: Climbing Trees}

% Describe, step by step, the project euler problem. Maybe make a
% more functional implementation of the solution?

\subsection{Case Study: A Top-Down Parser}
% Implement FIRST, FOLLOW, and the parser on a simple grammar.

\section{Smalltalk}
\subsection{The Type Model}
% Begin with the type model because Smalltalk is very type pure.
% Give a brief and intuitive primer on object oriented programming.
% Give a primer on dynamic typing. I should probably do some reading
% on the implementation of Smalltalk and finish Part 1 of the book.

\subsection{Control Flow}
% Describe how to use methods to do conditionals and iteration.

\subsection{Case Study: A Simple Text Adventure Interpreter}
% Describe the type model of the interactive fiction world

\subsection{The Parser Revisited}
% Adapt the SETL top-down parser into a more object-oriented style}

\section{Agda}

\subsection{Ordinary Types}
% Describe Haskell-style types using the Agda syntax.}

\subsection{Dependent Types}
% Do some example of a dependent type with a couple functions 
% ver it

\subsection{Pattern Matching Heaven}
% Show the mechanics of Agda's neat pattern matching system}

\subsection{Case Study: A Lambda Expression Evaluator}
% Implement a lambda expression evaluator using a type that depends
% on the number of free variables?

\subsection{The Parser a Final Time}
% Implement the parser again, this time using dependent types}

\section{Summary and Final Comparison}
% Offer final thoughts about the advantages and disadvantages
% of the languaes, and speculate how these advantages could be combined.

\end{document}
