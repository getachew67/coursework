\documentclass{article}

\title{Essential System Administration, Chapter 1}
\author{Jay R Bolton}

\addtolength{\oddsidemargin}{-.875in}
\addtolength{\evensidemargin}{-.875in}
\addtolength{\textwidth}{1.75in}
\addtolength{\topmargin}{-.875in}
\addtolength{\textheight}{1.75in}

\begin{document}

\maketitle

\section*{Reading Summary}
This book provides no problems or labs, so I'll simply write reading summaries
and then describe any new configurations that I set up in the lab based on the
chapter. In my summaries I will skip all the stuff I already know or do not
find interesting.

The key aspects of administration are:
\begin{itemize}
\item Know how things work.
\item Plan before doing.
\item Make everything reversible.
\item Make all changes incremental.
\item Test, test, test.
\end{itemize}

When modifying configs, do something like the following: \texttt{cp inittab inittab.dist ; chmod a-w inittab.dist}.

The author really stresses revision control, which we have set up at \texttt{svn://ocelot/ac/oslab}.

\subsection*{Super Users}

\texttt{su -} to get a full login session with \texttt{su}. \texttt{su -c cmd} to run only one command. 

Sudo is where the action is at. \texttt{/etc/sudoers} has the following syntax:

\begin{verbatim}
Host_Alias NAME1 = host1, host2
User_Alias NAME2 = user1, user2
Cmnd_Alias NAME3 = host1, host2

usern    hostn = cmnd : cmnd2
\end{verbatim}

\subsection*{Communication with Users}

\texttt{write username} sends a message. \texttt{talk} is a fancier visual version. And \texttt{wall} is for being even more obnoxious about it (sends message to all users). 

\subsection*{Plod}

The author plugs a program called \texttt{plod} that allows you to log what you
are working on. \texttt{plod [text]} places a timestamped log message in
~/.logdir/yyyymm. I thought that was a neat little thing.

\section*{Chapter 8---DNS}

I also read the first part of chapter 8, on managing DNS servers, to give me
the know-how to set up BIND on osl-share. 

DNS is divided into hierarchical `zones,' where each zone provides resolution
for lookups inside that zone. DNS responses to queries can be `recursive,'
meaning that DNS servers can automatically forward your query to another
server. 

I ended up installing bind with pacman and configured \texttt{/etc/named.conf}
to add the \texttt{slacr.acc} zone, whose configuration I put in
\texttt{/var/named/slacr.acc.zone}, which lists the addresses for the seven
machines in the oslab office. I will continue to study this chapter in order to
set up dynamic resolution for the computer center machines.

\end{document}
