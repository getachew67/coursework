\documentclass{article}
\title{Lab 4, Operating Systems}
\author{Jay R Bolton}

\begin{document}
\maketitle

I followed the instructions at the University of Waterloo website as
closely as I possibly could, referring to Peter's instructions for
any modifications. In general, it all went smoothly.

\subsection{Problems}

The only problem I had was that I complicated and confused the directory structure on my first installation attempt. I fixed this by adopting the following scheme:

\begin{enumerate}
\item \texttt{/opt/os161} : contains directories with the source code
for os161 as well as its tools (gcc, gdb, etc). It also contains the
  OS's root directory, so the OS is run out of
  \texttt{/opt/os161/root/}.
\item \texttt{/root/sys161} : contains the binaries and all the crap
that the various make procedures wrote out. This directory I used in
the \texttt{--prefix=} configuration argument.
\end{enumerate}
\end{document}
