\documentclass{article}

\usepackage{amsmath}


\title{Assignment 2, Data Structures}
\author{Jay R Bolton}

\begin{document}
\maketitle

2.1, 2.2, 2.6, 2.10, 2.14, 2.22

\begin{enumerate}
\item[\textbf{2.1}]

$$
2/N,37,\sqrt{N},N,NloglogN,NlogN,Nlog(N^2),Nlog^2N,N^{1.5},N^2,N^2logN,2^{N/2},2^N
$$

\item[\textbf{2.2}]

A is true.

\item[\textbf{2.6}]
 \begin{enumerate}
  I have found the exact complexities. I first turned each problem into a
  series of nested summations. Then I reduced them starting from the inside. I
  used the rules in the book for sums of squares and cubes, and had to google
  the sum of fourths.
  \item[\textbf{(1)}]

   $\Theta(N)$

   \begin{verbatim}
     int one(int n) {
       int i, sum=0;
       for(i = 0; i < n; i++) {
         sum++;
       }
       return sum;
     }
   \end{verbatim}

  \item[\textbf{(2)}]
   $\Theta(N^2)$

   \begin{verbatim}
     int two(int n) {
       int i, j, sum=0;
       for (i=0; i<n; i++)
         for (j=0; j<n; j++)
           sum++;
       return sum;
     }
   \end{verbatim}

  \item[\textbf{(3)}]
   $\Theta(N^3)$

   \begin{verbatim} 
     int three(int n) {
       int i, j, sum=0;
       for (i=0; i<n; i++)
         for (j=0; j<n*n; j++)
           sum++;
       return sum;
     }
   \end{verbatim}

  \item[\textbf{(4)}]
   $\Theta\left(\frac{N(N-1)}{2}\right)$

   \begin{verbatim} 
     int four(int n) {
       int i, j, sum=0;
       for (i=0; i<n; i++)
         for (j=0; j<i; j++)
           sum++;
       return sum;
     }
   \end{verbatim}

  \item[\textbf{(5)}]
   $\Theta\left(\frac{6(n-1)^5+ 15(n-1)^4 + 10(n-1)^3 - n+1 - 10n^3 +15n^2 -5n}{60}\right)$

   This is derived from the sum:
   $\sum^{n-1}_{i}\left(\sum_{j}^{i^2-1}\left(
     \sum_{k}^{j-1}1\right)\right)$

   \begin{verbatim} 
     int five(int n) {
       int i, j, k, sum=0;
       for (i=0; i<n; i++)
         for (j=0; j<i*i; j++)
           for (k=0; k<j; k++)
             sum++;
       return sum;
     }
   \end{verbatim}

  \item[\textbf{(6)}]
   $\Theta\left(\frac{(3i^4-10i^3+9i^2-2i)}{24}\right)$

     This was (painstakingly) calculated from the sum:
     $\sum^{n-1}_{i}\left(\sum_{j}^{i-1}\left(
       \sum_{k}^{i*j-1}1\right)\right)$

   \begin{verbatim}
    int six(int n) {
      int i, j, k, sum=0;
      for(i=1; i<n; i++)
        for(j=1; j<i*i; j++)
          if(j%i == 0) 
            for(k=0; k<j; k++) 
              sum++;
      return sum;
    }
   \end{verbatim}

 \end{enumerate}

\item[\textbf{2.10}]

 \begin{enumerate}
  \item[\textbf{a.}]
   $a_i = [2,1,8,4], x = 3, f(x) = 4x + 8x + x + 2$
   \begin{verbatim}
   poly = 0
   poly = 3 * 0 + 4 = 4
   poly = 3 * 4 + 8 = 20
   poly = 3 * 20 + 1 = 61
   poly = 3 * 61 + 2
   = 185
   \end{verbatim}
  \item[\textbf{b.}]
   It continually factors out x from left to right:
 
   $f(x) = 4x + 8x + x + 2 = ((4x + 8)x + 1)x + 2$
 
  \item[\textbf{b.}]
   $\Theta(N)$

   (Counting the number of multiplications as the measurement of running time)

 \end{enumerate}
 
 \item[\textbf{2.14}]
  If we measure our running time by the number of times we cross out a number, then we do:

  $f(n) = n/2 + n/3 + n/5 + n/7 ... n/q$

  Where q is the last prime less than or equal to the square root of n. Factoring out n, we get:

  $f(n) = n(1/2 + 1/3 + 1/5 + 1/7 ... 1/q)$

  In Euler's proof that the sum of the harmonic primes diverges (I looked it up
  on wikipedia), it states that the asymptotic upper bound of the harmonic
  primes up to n is loglogn. Thus, it seems like we can get a fairly tight
  upper bound with:

  $f(n) \epsilon \mathcal{O}(nloglogn)$


 \item[\textbf{2.22}]
  No. In the case where we are searching for an element that is greater than
  the maximum element of the list, we will loop endlessly on the second to last
  element. 

\end{enumerate}
\end{document}
