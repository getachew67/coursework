\documentclass{article}
\usepackage{amsmath}

\title{Assignment 1, Data Structures}
\author{Jay R Bolton}

\begin{document}
\maketitle

\begin{enumerate}

\item[\textbf{1.6}]
 \begin{enumerate}
  \item[\textbf{a.}]
   \begin{align}
    S &= 1 + \frac{1}{4} + \frac{1}{4^2} \cdots \frac{1}{4^i} \cdots \\
    4S &= 4 + 1 + \frac{4}{4^2} + \frac{4}{4^3} \cdots \frac{4}{4^i} \cdots \\
    &= 4 + 1 + \frac{1}{4} + \frac{1}{4^2}  \cdots \frac{1}{4^i} \cdots \\
    4S - S &=  4 = 3S \\
    \frac{3S}{3} &= S = \frac{4}{3}
   \end{align}
  \item[\textbf{b.}]
   \begin{align}
    S &= \frac{1}{4^1} + \frac{2}{4^2} \cdots \frac{i}{4^i} \cdots \\
    4S &= 1 + \frac{8}{4^2} + \frac{12}{4^3} \cdots \frac{4*i}{4^i} \cdots \\
    &= 1 + \frac{1}{2} + \frac{3}{4^2} + \frac{4}{4^3} \cdots \frac{i}{4^{i-1}} \cdots \\
    4S - S &= 1 + \frac{1}{4} + \frac{1}{4^2} + \frac{1}{4^3} \cdots \frac{1}{4^i} \cdots \\
    &= \frac{4}{3} \\
    \frac{3S}{3} &= S = \frac{4}{9}
   \end{align}
  \item[\textbf{c.}]
   \begin{align}
    S &= \frac{1^2}{4^1} + \frac{2^2}{4^2} \cdots \\
    4S &= 2 + \frac{3^2}{4^2} + \frac{4^2}{4^3} \cdots \\
    4S - S &= 3S = \frac{7}{4} + \frac{5}{4^2} + \frac{7}{4^3} \cdots\\
    4*3S &= \frac{33}{4} + \frac{7}{4^2} + \frac{9}{4^3} \cdots \\
    4*3S - 3S &= \frac{13}{2} + \frac{2}{4^2} + \frac{2}{4^3} \cdots \\
              &= \frac{13}{2} + 2(\frac{1}{4^2} + \frac{1}{4^3}\cdots) \\
              &= \frac{13}{2} + 2(\frac{4}{3} - \frac{5}{4}) \\
              &= \frac{13}{2} + 2(\frac{1}{12}) \\
              &= \frac{13}{2} + \frac{1}{6} \\
              &= \frac{39}{6} + \frac{1}{6} \\
              &= \frac{20}{3} \\
    9S &= \frac{20}{3} \\
    S &= \frac{20}{27}
   \end{align}
  \item[\textbf{d.}]
   This one was a real monkey of a problem. I did not really get a solution. I
   noted that with the powers, if we take the difference of their difference n
   times (where n is the power), then we end up with n! in the numerators. To
   take the difference of the differences of the numerators of S, we do $4S-S$
   and then $3(4S) - 3S$. We continue to do that n times.
   \begin{align}
    f(x,1) &= 4x-x \\
    f(x,n) &= f(4x-x,n-1) \\
    f(S) = ?
   \end{align}
 \end{enumerate}

\item[\textbf{1.10}]
 \begin{enumerate}
  \item[\textbf{a.}]
   Prove:
   \begin{align}
    \sum_{i=1}^n2i-1 &= n^2
   \end{align}
   Inductive hypothesis:
   \begin{align}
    \sum_{i=1}^n2i-1 = n^2 \rightarrow \sum_{i=1}^{n+1}2i-1 = (n+1)^2 \\
   \end{align}
   Induction:
   \begin{align}
    n^2 + 2(n+1) - 1 &= (n+1)^2 \\
    n^2 + 2n + 1 &= (n+1)^2 \\
    (n+1)^2 &= (n+1)^2 \\
   \end{align}
  \item[\textbf{b.}]
   Prove:
   \begin{align}
    \sum_{i=1}^ni^3 &= \left(\sum_{i=1}^ni\right)^2 \\
                    &= \left(\frac{n(n+1)}{2}\right)^2\\
                    &= \frac{n^2(n+1)^2}{4}
   \end{align}
   Base case:
   \begin{align}
    \frac{1^2(1+1)^2}{4} &= 1^3 \\
    \frac{1^2(1+1)^2}{4} &= 1 \\
    \frac{1(2)^2}{4}     &= 1\\
    \frac{4}{4}          &= 1\\
    1                    &= 1\\
   \end{align}
   Inductive hypothesis:
   \begin{align}
    \frac{n^2(n+1)^2}{4} \rightarrow &\frac{(n+1)^2(n+1)+1)^2}{4} \\
                         &= \frac{(n+1)^2(n+2)^2}{4}
   \end{align}
   Induction:
   \begin{align}
    \frac{n^2(n+1)^2}{4} + (n+1)^3 &= \frac{(n+1)^2(n+2)^2}{4} \\
    \frac{n^2(n+1)^2 + 4(n+1)^3}{4} &= \frac{(n+1)^2(n+2)^2}{4} \\
    \frac{(n+1)^2(n^2 + 4n+4)}{4} &= \frac{(n+1)^2(n+2)^2}{4} \\
    \frac{(n+1)^2(n+2)^2}{4} &= \frac{(n+1)^2(n+2)^2}{4}
   \end{align}
 \end{enumerate}

\end{enumerate}

\end{document}
