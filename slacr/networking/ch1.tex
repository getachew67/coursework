\documentclass{article}

\title{Networking, Chapter 1}
\author{Jay R Bolton}

\addtolength{\oddsidemargin}{-.875in}
\addtolength{\evensidemargin}{-.875in}
\addtolength{\textwidth}{1.75in}
\addtolength{\topmargin}{-.875in}
\addtolength{\textheight}{1.75in}

\begin{document}

\maketitle

\section*{Reading Summary}
This chapter provided a (very verbose) overview of all the material to come: a
general description of the internet, TCP and IP, and the concept of a protocol.
He went over some of the network technologies, like DSL, dial-up, and cable, as
well as the physical layer equipment like twisted pair copper wire, coax, etc.
He provided a detailed comparison of circuit switching versus packet switching
networks: packet switching has the advantage of greater overall usage of a
network while circuit switching provides uninterrupted connections with less
overall data transfer efficiency.

Various types of transmission delay were described. Nodal processing delay is
the time to actually process the header and determine the packet's destination.
When a process is queued at the gateway, it is also delayed. The transmission
delay happens while the packet is being pushed onto the link, and the
propogation delay is the actual time it takes to travel the distance between
the routers. The nodal delay of a process is a sum of the above four delays.
Packet loss occurs when a router's queue is maxed out. The end-to-end delay of
a packet is the sum of all the delays excluding the queueing delay.

The book then went into a general description of the five layers of the
Internet Protocol. The application layer includes HTTP, SMTP, FTP, and DNS. It
is generally concerned with the content of a message in a packet. Next, the
transport layer, which includes TCP and UDP, provides meta services like
delivery guarantee and packet segmentation. The network layer moves `datagrams'
using the IP protocol, which contains routing information and how information
for the hosts for acting on the datagrams in different ways. The link layer,
which is the lowest layer, provides logistics between the physical, local
network components.

\section*{Questions}
\begin{enumerate}

 \item[\textbf{R1.3}]
  A host stores and transmits data while the end system receives it.
 \item[\textbf{R1.5}]
  Cable, DSL, satellite... dsl is generally less expensive and slightly slower
  than cable, and satellite is generally very expensive.
 \item[\textbf{R1.11}]
  A circuit switched network provides a very reliable connection at the cost of
  having more unecessary downtime and less connection sharing. TDM provides
  more bandwidth (all available) but in brief slots of time.
 \item[\textbf{R1.12}]
  $L/R_2$
 \item[\textbf{R1.13}]
  Packet switching queues packets on demand; if you send the most packets, the
  packet switch is most likely to process your packets. TDM is not based on
  demand but on an even division of a length of time.
 \item[\textbf{R1.15}]
  Tier 1 ISPs serve the largest region and have the highest transmission rates.
  Tier 2 ISPs communicate with a subset of tier 2 ISPs and serve smaller
  regions.

\end{enumerate}

\section*{Problems}
\begin{enumerate}
 \item[\textbf{P1.1}]
  \begin{enumerate}
   \item
    Teller machine sends connection request to bank server.
   \item
    The bank server responds with a connection reply message indicating that it
    is okay to now send a request.
   \item
    The teller machine sends any of the following requests:
    \begin{enumerate}
     \item
      Authenticate: teller sends encrypted pass and card number. Bank responds
      with success or failure and possibly an html or interactive page of some
      sort with links to authenticated requests. The server also modifies its
      own state and creates an authenticated session for that user. On failure,
      responds with a failure page and server state is unmodified.
     \item
      Query balance: during an authenticated session, teller machine sends a
      query balance request, and the bank responds with a page listing the
      customer's balance. Without authentication, responds with a failure page.
     \item
      Withdraw: an authenticated session, teller machine sends a withdraw
      request. Server responds by changing the state of its database or whatnot
      for that user and sends a notification page. Without auth, only sends
      back a failure page. When the server-side code checks balance and finds
      an overdraft, then responds only with a failure page.
    \end{enumerate}
  \end{enumerate}
 \item[\textbf{P1.2}]
  \begin{enumerate}
   \item[\textbf{a}]
    $4n$
   \item[\textbf{b}]
    $n$
  \end{enumerate}
 \item[\textbf{P1.3}]
  \begin{enumerate}
   \item[\textbf{a}]
    It seems like a circuit switching network would be better, because both TDM
    and FDM multiplexing does not depend on network usage, but is pre-allocated
    according to a fixed scheme so would provide a very steady rate.
   \item[\textbf{b}]
    Even though the data rates are high enough, that does not account for down
    times or other failures that might cause queues of packets to back up. I'm
    not sure this is the answer he is looking for. 
  \end{enumerate}
  
\end{enumerate}

\section*{Lab Work}
I followed the book's first online lab, which involved installing and
configuring Wireshark and requesting and observing an HTTP packet inside
Wireshark.

\end{document}

\end{document}
