\documentclass{article}

\title{Distributed Algorithms, Chapter 1}
\author{Jay R Bolton}

\addtolength{\oddsidemargin}{-.875in}
\addtolength{\evensidemargin}{-.875in}
\addtolength{\textwidth}{1.75in}
\addtolength{\topmargin}{-.875in}
\addtolength{\textheight}{1.75in}

\begin{document}

\maketitle

\section*{Reading Summary}

The field of distributed computing is concerned with: interprocess
communication (shared memory, messaging), timing (synchronous or asynchronous),
and hardware failure. Distributed algorithms are complicated and have many
unknowns: the number of processors, the topology of the network, different
independent inputs, different timing and speeds, and so on.

Distributed algorithms are classified in this book by the `timing model.' This
includes synchronous execution, asynchronous, and partially synchronous. We
start with synchronous because it is the least complicated and can provide a
good introduction before going into the asynchronous model. Complexity in
distributed algorithms can be described as the amount of uncertainty in the
system, and synchronous models have the least.
\\[2em]
This chapter has no problems and was simply a comprehensive overview of all the
chapters following, so I only have this reading summary to offer.

\end{document}
