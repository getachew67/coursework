\documentclass{article}
\usepackage{multirow}
\title{Assignment 4, Operating Systems}
\author{Jay R Bolton}

\begin{document}
\maketitle

Ch 4 problems: 4.1, 4.2, 4.3, 4.4, 4.5, 4.7, 4.8, 4.11 (4.7 is a trick question, really just a setup for 4.8)

\begin{enumerate}

\item[\textbf{4.1}]
Yes. Any processor state information that needs to be saved and is shared among
the threads only needs to be saved once for all the threads.

\item[\textbf{4.2}]
If a ULT does a system call that manipulates the user address space, all
threads in that process will rely on that thread's system call.

\item[\textbf{4.3}]
\begin{enumerate}
\item[\textbf{a.}]
 Threads might lose some kernel level functionality that processes within
 sessions had.
\item[\textbf{b.}]
 Process level. If processes are replacing sessions, then processes are
 directly assigned the keyboard, mouse, and screen.
 \end{enumerate}

\item[\textbf{4.4}]
 Without the mapping, the ULT that does a system call would block all other
 threads within that process.

\item[\textbf{4.5}]
 No. 

\item[\textbf{4.7}]
\begin{enumerate}
\item[\textbf{a.}]
Nothing.
\item[\textbf{b.}]
No because the list is all negatives.
\end{enumerate}

\item[\textbf{4.8}]
If r is set in A but assigned to global positives in B, then if A and B are not
exactly synchronized, global positives will be incorrect.

\item[\textbf{4.11}]
\begin{enumerate}
\item[\textbf{a.}]
Possibly ULTs don't have to block the other ULTs in the same process if it does
a system call. Also we would not need to do a kernel level mode switch, I
guess. 
\item[\textbf{b.}]
LWP will go immediately to the running state upon wakeup. 
\item[\textbf{c.}]
A ULT is created and runs, which maps it to an LP that starts in the running
state. LWPs can be preempted and blocked and runs independently of the ULT.

\end{enumerate}


\end{enumerate}
\end{document}
