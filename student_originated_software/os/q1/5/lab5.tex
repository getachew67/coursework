\documentclass{article}
\title{Lab 5, Operating Systems}
\author{Jay R Bolton}

\begin{document}
\maketitle 

I painstakingly followed every step at Waterloo's Assignment 0 and
ran into no trouble at all. I was able to execute the print statement
inside main.c, run the debugger, and commit my OS161 source to a git
repository.

\section{Some notes}
\begin{itemize}
\item
The tricky part about \texttt{OPT\_A0} works if you follow Assignment 0 to the
letter. \texttt{OPT\_A0} is defined in the header file \texttt{opt-A0.h}, which comes
built-in to the OS source.

\item
I used GNU-Screen to manage a simultaenous debug and run window. You can start
GNU-Screen with \texttt{screen}, then rename the window with \texttt{C-a A} to
"run". Then start a second window with \texttt{C-a c} and rename that as
"debug." Swap back and forth with \texttt{C-a "}.

\item
I used git rather than CVS. In my OS161 source directory, I did a \texttt{git
init-db} which adds a \texttt{.git} directory. I then added all files with
\texttt{git add .}. To commit, you must set the git author environment
variables (you can see them by attempting to commit without having them set). I
set those variables (using \texttt{export}) in \texttt{~/.bashrc}

\end{itemize}
\end{document}
