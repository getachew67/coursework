\documentclass{article}
\usepackage{multirow}
\title{Assignment 6, Operating Systems}
\author{Jay R Bolton}

\begin{document}
\maketitle

Ch 6 problems: 6.1, 6.2, 6.4, 6.5, 6.6, 6.11, 6.14, 6.15, 6.18

\begin{enumerate}
\item{\textbf{6.1}}
1) \textbf{Mutual exclusion} Only car 1 can access a, car 2 b, car 3 c, and car
4 d.  2) \textbf{Hold and wait} Once car 1 consumes a and car 2 consumes b, car
1 will wait on the release of b.  3) \textbf{No preemption} Cars cannot bump
each other out of the way (in this model).  4) \textbf{Circular wait} Car 1
waits on car 2 which waits on car 3 which waits on car 4 which waits on car 1.

\item{\textbf{6.2}}
The OS (or traffic lights) could coordinate access based on columns and rows,
just like a real intersection: Car 1 and 3 drive until completion before the
other cars can go.

\item{\textbf{6.4}}
The graph gives four non-deadlock paths. Here is one: Process Q can get B while
process P can get A. Then process P releases A before process Q gets A. Then
process Q releases B before process P gets B. 

\item{\textbf{6.5}}
\begin{enumerate}
\item{\textbf{a}}
The available array is calculated by summing the columns and subtracting that
from total resource availability.

\item{\textbf{b}}
The need matrix is calculated by subtracting the cells in ``Current allocation"
from the corresponding cells in ``Maximum demand."

\item{\textbf{c}}
I'll represent the Available array as a Haskell-style list of tuples of resource ID and int.

\begin{enumerate}
\item P1 executes. ``Available" becomes: \texttt{[(A,4),(B,2),(C,3)(D,2)]}
\item P1 exits. ``Available" becomes \texttt{[A,6],[B,4],(C,6),(D,5)}
\item P2 executes. ``Available" becomes: \texttt{[A,3],[B,0],(C,2),(D,3)}
\item P2 exits. ``Available" becomes: \texttt{[A,10],[B,5],(C,6),(D,7)}
\item P3 executes. ``Available" becomes: \texttt{[A,8],[B,2],(C,3),(D,6)}
\item P3 terminates. ``Available" becomes: \texttt{[A,11],[B,5],(C,6),(D,8)}
\item P4 executes. ``Available" becomes: \texttt{[A,7],[B,4],(C,4),(D,7)}
\item P4 terminates. ``Available" becomes: \texttt{[A,12],[B,6],(C,6),(D,8)}
\item P5 executes. ``Available" becomes: \texttt{[A,9],[B,2],(C,3),(D,5)}
\item P5 terminates. ``Available" becomes: \texttt{[A,13],[B,6],(C,7),(D,9)}
\item P0 executes and terminates
\end{enumerate}

\item{\textbf{d}}
If we're assuming the request is \emph{instead of} current request, then it's
fine.  When P5 terminates, it frees 4, 4, 4, 4, which is more than enough for
the others.

If we assume the request is \emph{in addition to} the current request, then
it won't work if it goes in the above order. If we do P0 before P4 then it'll
work. 

\end{enumerate}

\item{\textbf{6.6}}
\begin{enumerate}
\item[a.]
P0 will try to get C when P2 already has it. P1 will try to get B when P0
already has it. P2 will try to get D when P1 already has it. Circular deadlock.

\item[b.]
Move \texttt{get(C)} to the end in P2. P2 will hold for P1. P1 will wait for
P0. P0 will not wait for anybody and will terminating, releasing B for P1, P1,
then P1 will complete and release D for P2.

\end{enumerate}

\item{\textbf{6.11}}
Yes. 1, 2, 3.
\begin{enumerate}
\item[a.]
Yes, 1,2,3

\item[b.]
No.

\end{enumerate}

\item{\textbf{6.14}}
\begin{enumerate}
\item[a.]
Yes. 
\begin{tabular}{l l}
\textbf{foo()} & \textbf{bar()} \\ \hline
semWait(S); & \\
semWait(R); & semWait(R); \\
x++; & \\
semSignal(S); & \\
semSignal(R); & semWait(S); \\
semWait(S); &  \\
\end{tabular}

\item[b.]
Yes.

\end{enumerate}

\item{\textbf{6.15}}
\texttt{(2 1 6 5)}

\item{\textbf{6.18}}
\begin{enumerate}
\item[a.]
If there is at least one leftie, then there is a leftie to the right of some rightie.
If a leftie a lefty is to the right of a righty, then there are two cases:
  The lefty grabs the left fork first
    The righty cannot take the right fork and is blocked on both forks.
  The rightie grabs the right fork first
    The lefty cannot take the left fork and is blocked on both forks.
Lemma:
If a philosopher is blocked on both forks, then there is no circular dependence.
  If a philosopher is blocked on both forks then both adjacent forks are available.
  Both adjacent forks are available

\item[b.]


\end{enumerate}

\end{enumerate}

\end{document}
