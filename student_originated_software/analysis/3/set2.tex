\documentclass{article}
\usepackage{amsmath}

\title{Set 2 Homework, Analysis of Algorithms}
\author{Jay R Bolton}

\addtolength{\oddsidemargin}{-.875in}
\addtolength{\evensidemargin}{-.875in}
\addtolength{\textwidth}{1.75in}
\addtolength{\topmargin}{-.875in}
\addtolength{\textheight}{1.75in}

\begin{document}
\maketitle

\begin{itemize}
\item P 52: 3.1-1, 3.1-2
\item P 60: 3.2-1, 3.2-2 and Problems: 3-1, 3-3, 3-4
\item P 107 Problems: 4-1, 4-2, 4-4
\end{itemize}


\section*{Chapter 3}

\begin{enumerate}

\item[\textbf{3.1-1}]

  Prove: $max(f(n), g(n)) = \Theta(f(n) + g(n))$

  By theorem 3.1, in order for a func to be big-Theta, it should be both big-O
  and big-Omega.
  
	\begin{align*}
  & \mathcal{O} : \\
  & max(f(n), g(n)) \leq f(n) + g(n) \\
  & max(f(n), g(n)) = O(f(n) + g(n)) \\
  & \Omega : \\
  & 2 * max(f(n), g(n)) >= f(n) + g(n) \\
  & max(f(n), g(n)) >= f(n) + g(n) * 1/2 \\
  & max(f(n), g(n)) = \Omega(f(n) + g(n))
	\end{align*}

\item[\textbf{3.1-2}]

  Prove: $ (n + a)^b = Theta(n^b) $

  Similarly to 3.1-1, we need to prove that the RHS is both big-O and big-Omega
  of the LHS.
  
	\begin{align*}
  & \mathcal{O} : \\
  & Show: (n+a)^b \leq n^b * c & \text{for some constant c} \\
  & Where: b > 0  \\
  & Cases: \\
  & a \leq 0: \\
  & (n-a))^b < n^b \\
  & (n-a))^b = O(n^b) \\
  & a > 0: \\
  & n+a \leq n*a \\
  & (n+a)^b \leq (n*a)^b = n^b * a^b \\
  & (n+a)^b \leq n^b * a^b \\
  & (n+a)^b = O(n^b) & \text{with constant $a^b$ for $a > 0$} \\ \\
	\end{align*}
	\begin{align*}
  & \Omega : \\
  & a \geq 0: \\
  & (n + a) \geq n \\
  & (n + a)^b \geq n^b \\
  & (n + a)^b = \Omega(n^b) \\
  & a < 0: \\
  & (n - a) \geq n \cdot -a \\
  & (n - a)^b \geq (n \cdot -a)^b \\
  & (n - a)^b \geq n^b \cdot -a^b \\
  & (n - a)^b = \mathcal{O}(n^b) & \text{with constant $a^b$ for $a < 0$}
	\end{align*}

\item[\textbf{3.2-1}]

	\begin{align*}
  & Show: \\
	& \text{If f(n) and g(n) are monotonically increasing, then so are:} \\
  & f(n) + g(n): \\
    & f(n) \leq f(m) \\
    & g(n) \leq g(m) \\
    & f(n) + g(n) \leq f(m) + g(m) \\
  & f(g(n)): \\
    & f(n) \leq f(m) \\
    & g(n) \leq g(m) \\
    & f(g(n)) \leq f(g(m)) \\
    & * \text{Let: $g(n) = p$ and $g(m) = q$ } \\
    & * \text{We know that $p \leq q$ because it was stated that $g(n) \leq g(m)$} \\
    & * \text{We already said $f(n) \leq f(m)$ for all $n \leq m$, and that $p \leq q$} \\
    & * \text{Thus $f(p) \leq f(q)$, that is $f(g(n)) \leq f(g(m))$}
	\end{align*}

	\begin{align*}
  & Show: \\
  & \text{If $f(n)$ and $g(n)$ are nonnegative, then:} \\
  & \text{$f(n) \cdot g(n)$ is monotonically increasing} \\
  & Definitions: \\
   & * f(n) \leq f(m) \text{  forall } n \leq m \\
   & * g(n) \leq g(m) \text{  forall } n \leq m \\
   & * f(n) > 0 \text{  forall } n \\
   & * g(n) > 0 \text{  forall } n \\
  & Conclusions: \\
	 & * \text{Since $f(n)$ and $g(n)$ are monotonically increasing and only positive, then} \\
		 & \text{they will only be positively increasing.} \\
   & * f(n) \cdot g(n) \leq f(m) \cdot g(m) \text{  forall } n \leq m \\
	 & * \text{This holds true because increasing positive integers multiplied will still} \\
		 & \text{be increasing.}
	\end{align*}

\item[\textbf{3.2-2}]

	\begin{align*}
	& Prove: \\
		& a^{log(b,c)} = c^{log(b,a)} \\
		& \text{I assume we can use the equations above this one.} \\
		& Definition: q = b^y <=> log(b,q) = y \\
			& a^{log(b,c)} = c^{log(b,a)} \\
			& = log(c,a^{log(b,c)}) = log(b,a) \\
			& = log(b,c) * log(c,a) = log(b,a) \\
			& = log(c,a) = log(b,a) / log(b,c) \\
			& = log(c,a) = log(c, a) \\
		& \text{This used equations on p56 above the equation we proved.}
	\end{align*}

\item[\textbf{3-1}]

The following is a lemma that I'll use for this problem:

\begin{enumerate}

	\item[\textbf{a.}] Prove: $ k \geq d \rightarrow p(n) = \mathcal{O}(n^k) $

	\begin{align*}
	& Show: \sum_{i=0}^d a_i n^i \leq c \cdot n^k \text{    for some constant c} \\
	& \text{Let $a_m = max(a_i)$ } \\
	& \sum_{i=0}^d a_i n^i \leq (a_m d) \cdot n^d \leq (a_m d) \cdot n^k \\
	& \sum_{i=0}^d a_i n^i = \mathcal{O}(n^k) & \text{    with constant $(a_m \cdot d)$ }
	\end{align*}

	\item[\textbf{b.}] Prove: $ k \leq d \rightarrow p(n) = \Omega(n^k) $

	\begin{align*}
		& Show: \sum_{i=0}^d a_i n^i \geq c \cdot n^k \text{    with some constant c} \\
		& \sum_{i=0}^d a_i n^i \geq n^d \geq n^k \\
		& \sum_{i=0}^d a_i n^i = \Omega(n^k) & \text{    with constant $1$ }
	\end{align*}

	\item[\textbf{c.}] Prove: $ k = d \rightarrow p(n) = \Theta(n^k) $

  See proof in (a) and (b); by Theorem 3.1, $n^d$ is also $\Theta$.

	\begin{align*}
		& Show: \sum_{i=0}^d a_i n^i \geq c \cdot n^d \text{    with some constant c} \\
		& Also: \sum_{i=0}^d a_i n^i \leq e \cdot n^d \text{    with some constant e} \\
		& \sum_{i=0}^d a_i n^i \leq (a_m d) \cdot n^d  \\
		& \sum_{i=0}^d a_i n^i \geq n^d \\
	\end{align*}

	\item[\textbf{d.}] Prove: $ k > d \rightarrow p(n) = o(n^k) $

	\begin{align*}
		& Show: \sum_{i=0}^d a_i n^i < c \cdot n^k \text{    with some constant c} \\
		& \sum_{i=0}^d a_i n^i \leq (a_m d) \cdot n^d < (a_m d) \cdot n^k \\
	\end{align*}

	\item[\textbf{e.}] Prove: $ k < d \rightarrow p(n) = \omega(n^k) $

	\begin{align*}
		& Show: \sum_{i=0}^d a_i n^i > c \cdot n^k \text{    with some constant c} \\
		& \sum_{i=0}^d a_i n^i \geq n^d > n^k \\
	\end{align*}

\end{enumerate}

\item[\textbf{3-3}]

From largest to smallest:

\begin{align*}
& 2^{2^n} \\
& (n+1)! \\
& n! \\
& e^n \\
& n \cdot 2^n \\
& 2^n \\
& (3/2)^n \\
& (lg n)^{lg n} \\
& (lg n)! \\
& n^3 \\
& n^2 \\
& n lg n, lg(n!) \\
& n \\
& 2^{\sqrt{2 lg n}} \\
& (lg n)^2 \\
& lg n \\
& \sqrt{lg n} \\
& lg lg n \\
& 2^{lg \cdot n} \\
& (lg n)* \\
& n^{1/lg n} \\
\end{align*} 

Some more of them may be in equivalence classes...

\item[\textbf{3-4}]

\begin{enumerate}
\item[\textbf{a.}]

False by counterexample: $n$ and $n^2$

\item[\textbf{b.}]

False by counterexample: $n$ and $n^2$

\item[\textbf{c.}]

True: $f(n) \leq g(n) \text{  and  } lg(f(n)) \leq lg(g(n))$

\item[\textbf{d.}]

True: $f(n) \leq g(n) \text{  and  } 2^{f(n)} \leq 2^{g(n)}$

\item[\textbf{e.}]

True: $f(n) \leq f(n)^2 $

\item[\textbf{f.}]

True by transpose symmetry.

\item[\textbf{g.}]

False: $n^2 > c * (n/2)^2$

\item[\textbf{h.}]

True:

\begin{align*}
f(n) + o(n) \leq 2 \cdot f(n) \\
f(n) + o(n) \geq 1/2 \cdot f(n)
\end{align*} 

\end{enumerate}

\end{enumerate}

\section*{Chapter 4}
\begin{enumerate}

\item[\textbf{4-1}]

I'll use master theorem on all of them (because it's quick/easy), except for
the last.

\begin{enumerate}

\item[\textbf{a.}]

	\begin{align*}
	& a = 2 \\
	& b = 2 \\
	& f(n) = n^4 \\
	& f(n) = \Omega(n^{log_2^2 + e}) \\
	& 2 \cdot (n^4)/(2^4) \leq c \cdot n^4 \\
	& T(n) = \Theta(n^4)
	\end{align*}

\item[\textbf{b.}]

	\begin{align*}
	& a = 1 \\
	& b = 7/10 \\
	& f(n) = n \\
	& f(n) = \Omega(n^{log_{7/10}^{1} + e}) \\
	& e = 1 \\
	& 1 \cdot (n)/(7/10) \leq c \cdot n \\
	& T(n) = \Theta(n)
	\end{align*}

\item[\textbf{c.}]

	\begin{align*}
	& a = 16 \\
	& b = 4 \\
	& f(n) = n^2 \\
	& f(n) = \Theta(n^{log_4^{16}}) \\
	& T(n) = \Theta(n^2 lg n)
	\end{align*}

\item[\textbf{d.}]

	\begin{align*}
	& a = 7 \\
	& b = 3 \\
	& f(n) = n^2 \\
	& f(n) = \Omega(n^{log_3^7 + e}) \\
	& 7 \cdot (n^2)/(3^2) \leq c \cdot n^2 \\
	& T(n) = \Theta(n^2)
	\end{align*}

\item[\textbf{e.}]

	\begin{align*}
	& a = 7 \\
	& b = 2 \\
	& f(n) = n^2 \\
	& f(n) = \mathcal{O}(n^{log_2^7 - e}) \\
	& T(n) = \Theta(n^{log_2^7})
	\end{align*}

\item[\textbf{f.}]

	\begin{align*}
	& a = 2 \\
	& b = 4 \\
	& f(n) = \sqrt{n} \\
	& f(n) = \Theta(n^{log_4^2}) \\
	& T(n) = \Theta(n^{1/2} lg n)
	\end{align*}

\item[\textbf{g.}]

This recurrence can be represented by the sum:

	\begin{align*}
	& c \cdot \sum_{i=0}^{\lfloor n/2\rfloor}n^2 \\
	& = c \cdot \frac{(n^2/4 + 1)(n + 1)}{6} \\
	& = c \cdot\frac{n^3/4 + n^2/4 + n + 1}{6} \\
	& = c \cdot (n^3 \cdot 1/24 + n^2 \cdot 1/24 + n/6 + 1/6 )\\
	& = \Theta(n^3)
	\end{align*}

\end{enumerate}

\item[\textbf{4-2}]

\begin{enumerate}

\item[\textbf{a.}]

The recursive representation of the complexity of binary sort is: $T(n) = T(n/2) + 1$

For case 1, our complexity is $T(lg n)$ by the master theorem.

For case 2, our recurrence is $T(n') = T(n'/2) + n$ and our complexity for this is $T(n lg n)$.

For case 3, our complexity is still $T(n lg n)$.

\item[\textbf{b.}]

For case 1, $T(n) = 2T(n/2) + n$ which is $\Theta(n lg n)$ by the master theorem.

For case 2, $T(n') = 2T(n'/2) + 4n$ which is $\Theta(n lg n)$ by the master theorem.

For case 3 $T(n) = 2T(n/2) + 4n$ which is $\Theta(n lg n)$ by the master theorem.

\end{enumerate}

\item[\textbf{4-4}]

Show: $\mathcal{F}(z) = z + z \mathcal{F}(z) + z^2 \mathcal{F}z$

\begin{align*}
& \mathcal{F}(z) = \sum_{i=0}^\infty \mathcal{F}_i z^i \\
& = z + \sum_{i=2}^\infty (\mathcal{F}_{i-1} + \mathcal{F}_{i-2}) z^i \\
& = z + \sum_{i=2}^\infty \mathcal{F}_{i-1} z^i  + \sum_{i=2}^{\infty} \mathcal{F}_{i-2} z^i \\
& = z + z \sum_{i=0}^\infty \mathcal{F}_{i} z^i  + z^2 \sum_{i=0}^{\infty} \mathcal{F}_{i} z^i \\
& = z + z \mathcal{F}(z) + z^2 \mathcal{F}(z) 
\end{align*}

Only had time to figure out the first part of this problem.

\end{enumerate}

\end{document}
