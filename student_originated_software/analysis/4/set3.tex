\documentclass{article}
\usepackage{amsmath}

\title{Set 3 Homework, Analysis of Algorithms}
\author{Jay R Bolton}

\addtolength{\oddsidemargin}{-.875in}
\addtolength{\evensidemargin}{-.875in}
\addtolength{\textwidth}{1.75in}
\addtolength{\topmargin}{-.875in}
\addtolength{\textheight}{1.75in}

\begin{document}
\maketitle

\begin{itemize}
\item p 166: 6.5-6
\item p 167: 6-1,6-2
\item p 178: 7.2-1, 7.2-5
\item p 180: 7.3-1
\item p 284: 7.4-2
\item p 185: 7-2, 7-4
\end{itemize}

\section*{Chapter 6}

\begin{enumerate}

\item[\textbf{6.5-6}]

Do `exchange' in `Heap-Increase-Key' with one assignment.

The original:

\begin{align*}
& HeapIncreaseKey(A, i, key): \\
& \ \ if\ key < A[i] \\
& \ \ \ \ error \ \text{``new key is smaller than current key''} \\
& \ \ A[i] = key \\
& \ \ while\ i > 1\ and\ A[Parent(i)] < A[i] \\
& \ \ \ \ exchange\ A[i]\ with\ A[Parent(i)] \\
& \ \ I = Parent(i)
\end{align*}

With three assignments:

\begin{align*}
& HeapIncreaseKey(A, i, key): \\
& \ \ if\ key < A[i] \\
& \ \ \ \ error \ \text{``new key is smaller than current key''} \\
& \ \ A[i] = key \\
& \ \ while\ i > 1\ and\ A[Parent(i)] < A[i] \\
& \ \ \ \ tmp = A[i] \\
& \ \ \ \ A[i] = A[Parent(i)] \\
& \ \ \ \ A[Parent(i)] = tmp \\
& \ \ \ \ i = Parent(i)
\end{align*}

With one assignment:

\begin{align*}
& HeapIncreaseKey(A, i, key): \\
& \ \ if\ key < A[i] \\
& \ \ \ \ error \ \text{``new key is smaller than current key''} \\
& \ \ while\ i > 1\ and\ A[Parent(i)] < key \\
& \ \ \ \ A[i] = A[Parent(i)] \\
& \ \ \ \ i = Parent(i) \\
& \ \ A[i] = key \\
\end{align*}

That was a real fun little puzzle.

\item[\textbf{6-1}]

	\begin{enumerate}

	\item[\textbf{(a)}]
	No. The counterexample is $[N,1,2,3]$. BMH produces $[N,3,2,1]$ while BMH' produces $[N,3,1,2]$. Both are heaps.

	\item[\textbf{(b)}]
	Max-Heap-Insert requires $\Theta(lg\ n)$ time. In Build-Max-Heap', we are
	looping that function $n-1$ times. Everything else is constant, so our bound is
	$\Theta(n\ lg\ n)$.

	\end{enumerate}

\item[\textbf{6-2}]


	\begin{enumerate}

	\item[\textbf{(a)}]
	Same way, but you'd have to store or pass d and the children would be
	calculated with $di + 1$ through $di+d$ where `i' is the current index.

	\item[\textbf{(b)}]
	The height would be $log_d(n)$.

	\item[\textbf{(c)}]
	
	\begin{align*}
	& \text{ExtractMax}(A) \\
	& \ \ if\ A.\text{heapsize} < 1 \\
	& \ \ \ \ \text{error}\ ``heap\ underflow" \\
	& \ \ max = A[1] \\
	& \ \ A[1] = A[A.\text{heapsize}] \\
	& \ \ \text{MaxHeapify}(A,1) \\
	& \ \ \text{return}\ max
	\end{align*}
	
	\begin{align*}
	& \text{MaxHeapify}(A,i) \\
	& \ \ largest = i \\
	& \ \ for\ c = di + 1\ upto\ di + d \\
	& \ \ \ \ if\ c \leq A.\text{heapsize}\ and\ A[c] > A[largest] \\
	& \ \ \ \ \ \ largest = c \\
	& \ \ if\ largest \neq i \\
	& \ \ \ \ exchange\ A[i]\ with\ A[largest] \\
	& \ \ \ \ \text{MaxHeapify}(A,largest) 
	\end{align*}

	ExtractMax remains unchanged, but MaxHeapify must now loop d times through
	all subtrees. Its complexity will be $\mathcal{O}(log_bn)$

	\item[\textbf{(d, e)}]
	Both Insert and IncreaseKey can be implemented the same since neither depend
	on the selection of children.

	\end{enumerate}

\item[\textbf{6-3}]

	\begin{enumerate}
	\item[\textbf{(a)}]
	\[ \begin{matrix}
	2 & 3 & 4 & 5 \\
	8 & 9 & 12 & 14 \\
	16 & \infty & \infty & \infty
	\end{matrix} \]

	\item[\textbf{(b)}]

	Y[1,1] will be the least element in the matrix (least of the least of the
	columns and least of the least of the rows). If Y[1,1] is infinity/null, then
	there is no least element.

	If Y[1,1] contains a non-null element then that means we have a least
	element. We have at least one element in that case, where m and n are 1.

	\end{enumerate}


\end{enumerate}

\section*{Chapter 7}

\begin{enumerate}

\item[\textbf{7.2-1}]

Prove: $T(n) = T(n-1) + \Theta(1)$ has complexity $\Theta(n^2)$.

	\begin{align*}
	\text{Inductive Hypothesis: } T(n) \leq c \cdot n^2 \\
	\text{Also: } T(n) \geq c \cdot n^2 \\
	\text{Induction: } \\
	T(n) \leq b(n-1)^2 + n^2 \text{ for some constant b }\\
	\leq bn^2 + n^2 \\
	= (b+1)n^2 \\
	\leq c \cdot n^2 \\
	T(n) \geq b(n-1)^2 + n^2 \text{ for some constant b }\\
	\geq n^2 \\
	= 1 \cdot n^2 \\
	= c \cdot n^2
	\end{align*}

\item[\textbf{7.2-5}]


For the minimum depth, our recurrence is $T(n) = 2T(n/2) + n$. The proportion will be one-half to one-half, which is our best case.

For the maximum depth, our recurrence is $T(n) = 2T(n-1) + n$. The proportion
in this case is $\frac{n-1}{n}$ to $\frac{1}{n}$. This is the worst case.

The height of the minimum depth is $lg_2 n$ and the height of the maximum depth is $n$.

\item[\textbf{7.3-1}]
Because the worst case may be asymptomatic and the randomized version closer to average.

\item[\textbf{7.4-2}]

By induction with hypothesis: $T(n) \geq c \cdot n\ lg\ n$

The best-case recurrence is $T(n) = 2T(n/2) + n$, where each subproblem is
partitioned evenly in two.

	\begin{align*}
	T(n) \geq 2(n/2)lg(n/2) + dn \\
	= n\ lg(n/2) + dn \\
	= n\ lg\ n - n\ lg\ 2 + dn \\
	= n\ lg\ n - n + dn \\
	\geq c \cdot n\ lg\ n \\
	\end{align*}

\item[\textbf{7-2}]

	\begin{enumerate}

	\item[\textbf{(a)}]

	With all elements equal, the running time would be worst case at $\mathcal{O}(n^2)$

	\item[\textbf{(b)}]

	Rewrite partition so that all elements equal to the pivot are in the middle.

	\begin{verbatim}
	partition(a)
	
	\end{verbatim}

	\item[\textbf{(a)}]
	\item[\textbf{(a)}]

	\end{enumerate}

\item[\textbf{7-4}]

\end{enumerate}

\end{document}
